\documentclass[11pt,a4paper]{article}

% --- Paquetes básicos ---
\usepackage[spanish,es-nodecimaldot,shorthands=off]{babel}
\usepackage[utf8]{inputenc}
\usepackage[T1]{fontenc}
\usepackage{lmodern}
\usepackage[a4paper,margin=2.5cm]{geometry}
\usepackage{microtype}
\usepackage{enumitem}
\usepackage{hyperref}
\hypersetup{colorlinks=true,linkcolor=blue,urlcolor=blue}
\usepackage{titlesec}
\usepackage{xcolor}
\usepackage{tcolorbox}
\usepackage{graphicx}

% --- TikZ para el diagrama ---
\usepackage{tikz}
\usetikzlibrary{arrows.meta,positioning,fit,shapes.misc}

% --- Estilos ---
\definecolor{accent}{HTML}{2E86AB}
\titleformat{\section}{\Large\bfseries\color{accent}}{\thesection.}{0.5em}{}
\titleformat{\subsection}{\large\bfseries}{\thesubsection}{0.5em}{}

\setlist[itemize]{leftmargin=1.4em}
\setlist[enumerate]{leftmargin=1.6em}

% --- Portada simple ---
\title{\textbf{Sistema Big Data para Interacciones con Películas}\\[2mm]
\large Descripción de proyecto, arquitectura y componentes}
\author{Equipo de Proyecto}
\date{\today}

\begin{document}
\maketitle
\vspace{-0.5em}
\hrule
\vspace{1.0em}

\section{DESCRIPCIÓN DEL PROYECTO}

\subsection*{Objetivo Central}
Analizar y visualizar en tiempo real las interacciones de usuarios con películas mediante un \textit{pipeline} completo de Big Data que combina procesamiento en \textbf{streaming} para métricas inmediatas y procesamiento \textbf{batch} para análisis históricos, permitiendo la detección de tendencias y patrones de comportamiento.

\subsection*{Dataset Seleccionado}
\begin{itemize}
  \item \textbf{Nombre:} \textit{Movies Dataset}
  \item \textbf{Formato:} JSON
\end{itemize}

\subsection*{Volumen}
\begin{itemize}
  \item 100 películas con información completa.
  \item \textbf{Streaming continuo:} generación de 1 interacción cada 2 segundos (1{,}800 interacciones/hora).
  \item \textbf{Almacenamiento histórico:} todos los datos se persisten en HDFS para análisis batch.
  \item \textbf{Simulación de Big Data:} arquitectura escalable que puede manejar millones de registros.
\end{itemize}

\subsection*{Características}
\begin{itemize}
  \item \textbf{Atributos estructurados:} ID, nombre, género, \textit{rating}, popularidad, descripción.
  \item \textbf{Temporalidad:} marcas de tiempo (\textit{timestamps}) precisas para cada interacción.
  \item \textbf{Etiquetas:} tipos de interacción (\texttt{click}, \texttt{view}, \texttt{rating}, \texttt{purchase}).
  \item \textbf{Datos limpios:} estructura consistente y validada.
  \item \textbf{Metadatos ricos:} información completa de películas para análisis multidimensional.
\end{itemize}

\subsection*{Pertinencia}
\begin{itemize}
  \item Relación directa con el objetivo de analizar comportamiento de usuarios.
  \item Datos realistas que simulan plataformas de \textit{streaming} reales.
  \item Estructura flexible que permite múltiples tipos de análisis.
  \item Escalabilidad demostrada para crecer en volumen y complejidad.
\end{itemize}

\subsection*{Diagrama del Pipeline}

\begin{center}
% Desactivar la shorthand "!" de babel para evitar conflictos con TikZ (p. ej. "accent!10")
\shorthandoff{!}
\begin{tikzpicture}[
  node distance=1.9cm and 1.2cm,
  box/.style={draw, rounded corners, minimum width=2.8cm, minimum height=0.9cm, align=center, thick, fill=gray!10},
  smallbox/.style={draw, rounded corners, minimum width=3.0cm, minimum height=0.9cm, align=center, thick, fill=gray!10},
  >={Stealth[length=2.2mm]},
  every node/.style={font=\small}
]
% Fila principal
\node[box, fill=accent!10] (producer) {Producer};
\node[box, right=of producer] (kafka) {Kafka};
\node[box, right=of kafka] (sparkredis) {Spark \& Redis};
\node[box, right=of sparkredis] (hdfs) {HDFS};
\node[box, right=of hdfs] (dash) {Dashboard};

\draw[->, thick] (producer) -- (kafka);
\draw[->, thick] (kafka) -- (sparkredis);
\draw[->, thick] (sparkredis) -- (hdfs);
\draw[->, thick] (hdfs) -- (dash);

% Rama inferior
\node[smallbox, below=1.2cm of sparkredis, xshift=-1.6cm] (yarn) {YARN \& MapReduce};
\node[smallbox, below=1.2cm of hdfs, xshift=1.6cm] (batch) {Batch \& Spark Jobs};

\draw[->, thick] (sparkredis.south) |- (yarn.north);
\draw[->, thick] (hdfs.south) |- (batch.north);
\end{tikzpicture}
\shorthandon{!}
\end{center}

\subsection*{Enfoque: Combinación Batch + Streaming}
Arquitectura \textbf{Lambda} que combina procesamiento en tiempo real para métricas inmediatas y procesamiento \textbf{batch} para análisis profundos.

\section{Componentes y Funciones}

\subsection*{Capa de Ingesta (Streaming)}
\begin{itemize}
  \item \textbf{Producer:} genera datos simulados de interacciones de usuarios cada 2 segundos.
  \item \textbf{Kafka:} sistema de mensajería distribuido que \emph{bufferiza} los datos en tiempo real.
\end{itemize}

\subsection*{Capa de Procesamiento (Streaming + Batch)}
\begin{itemize}
  \item \textbf{Spark Consumer:} procesa datos en tiempo real, calcula métricas y almacena en Redis.
  \item \textbf{Redis:} base de datos en memoria para métricas en tiempo real del \textit{dashboard}.
  \item \textbf{HDFS:} almacenamiento distribuido para todos los datos históricos.
\end{itemize}

\subsection*{Capa de Análisis (Batch)}
\begin{itemize}
  \item \textbf{Batch Processor:} ejecuta análisis periódicos sobre datos históricos en HDFS.
  \item \textbf{MapReduce con YARN:} procesamiento distribuido de grandes volúmenes de datos.
  \item \textbf{Spark Jobs:} análisis avanzados y agregaciones complejas.
\end{itemize}

\subsection*{Capa de Visualización}
\begin{itemize}
  \item \textbf{Dashboard:} interfaz web en tiempo real que muestra métricas actualizadas cada 3 segundos.
\end{itemize}

\vfill
\begin{tcolorbox}[colback=accent!5,colframe=accent,title=Resumen]
Este documento describe un \textit{pipeline} híbrido (Lambda) que integra \textbf{streaming} (métricas inmediatas con Kafka, Spark y Redis) y \textbf{batch} (análisis históricos con HDFS, YARN/MapReduce y \textit{Spark jobs}) para modelar y visualizar el comportamiento de usuarios en una plataforma de películas.
\end{tcolorbox}

\end{document}
